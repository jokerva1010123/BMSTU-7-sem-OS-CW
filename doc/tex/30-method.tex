\section{Технологический раздел}
В данном разделе выбирается язык программирования, на котором будет реализована поставленная задача, производится выбор среды разработки и рассматриваются некоторые мометы реализации загружаемого модуля ядра.
\subsection{Выбор языка программирования}
В качестве языка программирования для реализации данной курсовой работы был выбран язык C. При помощи этого языка реализованы все модули ядра и драйверы с использованием встроенного в ОС Linux компилятора GCC.
\subsection{Выбор среды разработки}
В качестве среды разработки был выбран стандартный тектовый редактор ОС Linux.

Ниже показано содержимое Makefile, содержащего набор инструкций, используемых утилитой make в инструментарии автоматизации сборки.

\begin{lstlisting}[caption = Содержимое Makefile, label = {lst:1}]
CONFIG_MODULE_SIG=n

ifneq ($(KERNELRELEASE),)
	obj-m += module_dev.o
else
	KERNELDIR ?= /lib/modules/$(shell uname -r)/build
	PWD := $(shell pwd)
default:
	make -C $(KERNELDIR) M=$(PWD) modules
clean:
	make -C $(KERNELDIR) M=$(PWD) modules clean
	rm -rf a.out
endif
\end{lstlisting}

\subsection{Описание некоторых моментов реализации}
\subsubsection{Модуль ядра}
Код функции, обрабатывающая запись в специальный файл.
\begin{lstlisting}[caption = , label = {lst:2}]
ssize_t module_dev_write(struct file *file, const char __user *buf, size_t size, loff_t *offset)
{
    printk(KERN_INFO "module_dev: called write %ld\n", size);

    copy_from_user(msg, buf, size);
    msg_length = size;

    is_empty = 0;

    if (msg[0] == '1')
    {
        tasklet_schedule(&my_tasklet);
    }
    printk(KERN_INFO "module_dev: write msg '%s'\n", msg);

    return size;
}
\end{lstlisting}

\subsubsection{Работа с индикаторами клавиатуры}
Доступ к индикаторам клавиатуры реализован через виртуальную консоль, доступную в ядре через библиотеку linux/console\_struct.h. Отложенное действие реализовано через тасклет.

Поставновка таймера на выполнение
\begin{lstlisting}[caption = , label = {lst:2}]
struct tty_driver *load_keyboard_led_driver(void)
{
    int i;
    struct tty_driver *driver;

    printk(KERN_INFO "kbleds: loading\n");
    printk(KERN_INFO "kbleds: fgconsole is %x\n", fg_console);
    printk(KERN_INFO "kbleds: MAX_NR_CONSOLES %i", MAX_NR_CONSOLES);
    
    for (i = 0; i < MAX_NR_CONSOLES; i++)
    {
        if (!vc_cons[i].d)
            break;
        printk(KERN_INFO "poet_atkm: console[%i/%i] #%i, tty %lx\n", i,
               MAX_NR_CONSOLES, vc_cons[i].d->vc_num,
               (unsigned long)vc_cons[i].d->port.tty);
    }

    printk(KERN_INFO "kbleds: finished scanning consoles\n");

    driver = vc_cons[fg_console].d->port.tty->driver;
    printk(KERN_INFO "kbleds: tty driver magic %x\n", driver->name);

    return driver;
}
\end{lstlisting}

Функция, включающая/выключающая индикаторы
\begin{lstlisting}[caption = , label = {lst:2}]
#define ALL_LEDS_ON 0x07
#define RESTORE_LEDS 0xFF

int is_led_on = 0;

void tasklet_fn_toggle_led(unsigned long data)
{
    unsigned long status;

    if (is_led_on)
    {
        is_led_on = 0;
        status = ALL_LEDS_ON;
    }
    else
    {
        is_led_on = 1;
        status = RESTORE_LEDS;
    }

    (my_driver->ops->ioctl)(vc_cons[fg_console].d->port.tty, KDSETLED, status);

}
\end{lstlisting}

\subsubsection{Программа пользователя}
\begin{lstlisting}[caption = , label = {lst:2}]
#include <alsa/asoundlib.h>
#include <stdio.h>
#include <stdlib.h>

void write_signal_to_proc()
{
    FILE *fp = fopen("/proc/module_dir/dev", "w");
    if (fp == NULL)
        printf("unable to open proc file\n");

    fprintf(fp, "1");
    fclose(fp);
}

snd_pcm_t *pcm_device;

int main()
{
    int count = 1;
    while(1)
    {
        int totalCards = 0;   
        int cardNum = -1;     
        int err;

        for (;;) {
            if ((err = snd_card_next(&cardNum)) < 0) {
                fprintf(stderr, "Can't get the next card number: %s\n",
                                snd_strerror(err));
                break;
            }
            if (cardNum < 0)
                break;
            ++totalCards;   
        }
        
        if(count < totalCards)
        {
            printf("%d %d\n", count, totalCards);
            printf("Headset connected\n");
            write_signal_to_proc();
            write_signal_to_proc();
        }
        else if(count > totalCards)
        {
            printf("%d %d\n", count, totalCards);
            printf("Headset disconnected\n");
            write_signal_to_proc();
            write_signal_to_proc();
        }
        count = totalCards;
        snd_config_update_free_global();
    }
}
\end{lstlisting}