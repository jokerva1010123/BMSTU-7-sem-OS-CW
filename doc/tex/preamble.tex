%----------------------- Преамбула -----------------------
\documentclass[ut8x, 14pt, oneside, a4paper]{extarticle}

%----------------------- Преамбула -----------------------

\usepackage{extsizes} % Для добавления в параметры класса документа 14pt

% Для работы с несколькими языками и шрифтом Times New Roman по-умолчанию
\usepackage[english,russian]{babel}
\usepackage[T2A]{fontenc}
\usepackage{fontspec}
\setmainfont{Times New Roman}
\usepackage{textcomp}

\usepackage{totcount}

\newtotcounter{citnum} %From the package documentation
\def\oldbibitem{} \let\oldbibitem=\bibitem
\def\bibitem{\stepcounter{citnum}\oldbibitem}
% ГОСТовские настройки для полей и абзацев
\usepackage[left=30mm,right=10mm,top=20mm,bottom=20mm]{geometry}
\usepackage{misccorr}
\usepackage{indentfirst}
\usepackage{enumitem}
\setlength{\parindent}{1.25cm}
%\setlength{\parskip}{1em} % поменять
%\linespread{1.3}

\usepackage{setspace}
\onehalfspacing
\usepackage{caption}
\setlist{nolistsep} % Отсутствие отступов между элементами \enumerate и \itemize

% Дополнительное окружения для подписей
\usepackage{array}
\newenvironment{signstabular}[1][1]{
	\renewcommand*{\arraystretch}{#1}
	\tabular
}{
	\endtabular
}
% Переопределение стандартных \section, \subsection, \subsubsection по ГОСТу;
% Переопределение их отступов до и после для 1.5 интервала во всем документе
\usepackage{titlesec}

\titleformat{\section}[block]
{\bfseries\normalsize\filcenter}{\thesection}{1em}{}

\titleformat{\subsection}[hang]
{\bfseries\normalsize}{\thesubsection}{1em}{}
\titlespacing\subsection{\parindent}{\parskip}{\parskip}

\titleformat{\subsubsection}[hang]
{\bfseries\normalsize}{\thesubsubsection}{1em}{}
\titlespacing\subsubsection{\parindent}{\parskip}{\parskip}

% Работа с изображениями и таблицами; переопределение названий по ГОСТу
%\renewcommand{\thefigure}{\thechapter.\arabic{figure}}

\usepackage{caption}

\captionsetup[figure]{labelsep=endash, singlelinecheck=false, justification=centering}
\captionsetup[table]{justification=raggedleft, singlelinecheck=false, labelsep=endash}
\addto\captionsrussian{\renewcommand{\figurename}{Рисунок}}
\usepackage{chngcntr}
\counterwithin{figure}{section}
\counterwithin{table}{section}
\counterwithin{equation}{section}
\renewcommand{\labelenumi}{\arabic{enumi}.}
\renewcommand{\labelenumii}{\arabic{enumii})}
\AddEnumerateCounter(\asbuk)
\renewcommand{\labelenumiii}{\asbuk{enumiii})}
\renewcommand{\labelitemi}{--}

%\renewcommand{\thetable}{\thechapter.\arabic{figure}}
\usepackage{graphicx}
%\usepackage{slashbox} % Диагональное разделение первой ячейки в таблицах

% Цвета для гиперссылок и листингов
\usepackage{color}

% Гиперссылки \toc с кликабельностью
\usepackage{hyperref}

\hypersetup{
	hidelinks
}

% Листинги
%\setsansfont{Arial}
%\setmonofont{Courier New}
\usepackage{color} % Цвета для гиперссылок и листингов
\definecolor{comment}{rgb}{0,0.5,0}
\definecolor{plain}{rgb}{0.2,0.2,0.2}
\definecolor{string}{rgb}{0.91,0.45,0.32}
\hypersetup{citecolor=blue}
\hypersetup{citecolor=black}
\newfontfamily\lstfont{Times New Roman}

\usepackage{listings}
\lstset{
	basicstyle=\fontsize{8}{10}\linespread{0.8}\lstfont,
	language=sql, % Или другой ваш язык -- см. документацию пакета
	commentstyle=\color{comment},
	numbers=left,
	numberstyle=\tiny,
	stepnumber=1,
	numbersep=5pt,
	xleftmargin =.19in,
	tabsize=4,
	extendedchars=\true,
	breaklines=true,
	keywordstyle=\color{blue},
	frame=single,
	stringstyle=\lstfont\color{string}\lstfont,
	showspaces=false,
	showtabs=false,
	showstringspaces=false,
%	xleftmargin=17pt,
%	framexleftmargin=17pt,
%	framexrightmargin=5pt,
%	framexbottommargin=4pt,
	showstringspaces=false,
	inputencoding=utf8x,
	keepspaces=true,
	captionpos=t,
	breakatwhitespace=false,
}
%\lstset{
%	language = python,
%	basicstyle=\small\sffamily,
%	numbers=left,
%	numberstyle=\tiny,
%	stepnumber=1,
%	numbersep=5pt,
%	xleftmargin =.19in,
%	showspaces=false,
%	showtabs=false,
%	frame=single,
%	tabsize=2,
%	captionpos=t,
%	breaklines=true,
%	breakatwhitespace=false,
%	escapeinside={\#*}{*)}
%}

%\DeclareCaptionLabelSeparator{line}{\ --\ }
\DeclareCaptionFont{white}{\color{white}}
\captionsetup[lstlisting]{
	singlelinecheck=false,
	justification=raggedright,
	labelsep=endash,
}

\usepackage{ulem} % Нормальное нижнее подчеркивание
\usepackage{hhline} % Двойная горизонтальная линия в таблицах
\usepackage[figure,table]{totalcount} % Подсчет изображений, таблиц
\usepackage{rotating} % Поворот изображения вместе с названием
\usepackage{lastpage} % Для подсчета числа страниц

\makeatletter
\renewcommand\@biblabel[1]{#1.}
\makeatother


\usepackage{color}
\usepackage[cache=false, newfloat]{minted}
\newenvironment{code}{\captionsetup{type=listing}}{}
\SetupFloatingEnvironment{listing}{name=Листинг}
\usepackage{float}
\usepackage{amsmath}
%\usepackage{slashbox}